\xname{predefined}
\chapter{Predefined Analyses}
\label{chap:predefined}

Petablox provides many standard analyses.
This chapter first explains how to run any such analysis and then provides
descriptions of various predefined analyses.

\section{Running an Analysis}
\label{sec:running-predefined}

Each analysis in Petablox is written modularly, independent of other analyses,
along with lightweight annotations specifying the inputs and outputs of the
analysis.  Petablox's runtime automatically computes dependencies between analyses
(e.g., determines which analysis produces as output a result that is needed as
input by another analysis).  Before running a desired analysis, Petablox
recursively runs other analyses until the inputs to the desired analysis have
been computed; it finally runs the desired analysis to produce the outputs of
that analysis.

Each analysis in Petablox has a unique name that can be used to run the
analysis from the command-line.  You can obtain the names of all analyses
predefined in Petablox by running the following command:

\begin{framed}
\begin{verbatim}
ant -Dpetablox.out.dir=<OUT_DIR> -Dpetablox.print.project=true run
\end{verbatim}
\end{framed}

This command produces files \code{targets_sortby_name.html}, \code{targets_sortby_kind.html},
and \code{targets_sortby_producers.html} in directory \code{<OUT_DIR>}.  These
files publish the names and descriptions of all analyses that are predefined in
Petablox, along with the inputs and outputs of each analysis.

The following command runs the analysis
named \code{<ANALYSIS_NAME>} on the program specified by directory \code{<WORK_DIR>}
(see Chapter \ref{chap:setup} for how to setup a program):

\begin{framed}
\begin{verbatim}
ant -Dpetablox.work.dir=<WORK_DIR> -Dpetablox.run.analyses=<ANALYSIS_NAME> run
\end{verbatim}
\end{framed}

For instance, the following command runs a basic points-to and call-graph
analysis (called 0CFA) provided in Petablox:

\begin{framed}
\begin{verbatim}
ant -Dpetablox.work.dir=<WORK_DIR> -Dpetablox.run.analyses=cipa-0cfa-dlog run
\end{verbatim}
\end{framed}

This instructs Petablox to run the analysis named \code{cipa-0cfa-dlog},
which is defined in file \code{main/src/petablox/analyses/alias/cipa_0cfa.dlog}.

The output of the above command is of the form:

\begin{framed}
{\small
\begin{verbatim}
Buildfile: build.xml

run:
     [java] Petablox run initiated at: Mar 13, 2011 10:31:08 PM
     [java] ENTER: cipa-0cfa-dlog
     [java] ENTER: T
     [java] ENTER: RTA
     [java] Iteration: 0
     [java] Iteration: 1
     [java] Iteration: 2
     [java] LEAVE: RTA
     [java] SAVING dom T size: 1386
     [java] LEAVE: T
     [java] ENTER: F
     [java] SAVING dom F size: 4120
     [java] LEAVE: F
     ...
     [java] ENTER: MputStatFldInst
     [java] SAVING rel MputStatFldInst size: 739
     [java] LEAVE: MputStatFldInst
     [java] ENTER: statIM
     [java] SAVING rel statIM size: 3359
     [java] LEAVE: statIM
     [java] Starting command: 'java ... petablox_analyses_alias_cipa_0cfa.dlog'
     [java] Relation VH: 541 nodes, 449.0 elements (V0,H0)
     [java] Relation FH: 137 nodes, 8.0 elements (H0,F0)
     [java] Relation HFH: 199 nodes, 35.0 elements (H0,F0,H1)
     [java] Relation IM: 590 nodes, 69.0 elements (I0,M0)
     [java] Finished command: 'java ... petablox_analyses_alias_cipa_0cfa.dlog'
     [java] LEAVE: cipa-0cfa-dlog
     [java] Petablox run completed at: Mar 13, 2011 10:31:36 PM
     [java] Total time: 00:00:27:671 hh:mm:ss:ms

BUILD SUCCESSFUL
Total time: 28 seconds
\end{verbatim}
}
\end{framed}

The 0CFA analysis consumes and produces multiple {\it program relations}.  The
consumed program relations include \code{MputStatFldInst} and \code{statIM},
each of which is produced by a separate analysis with the corresponding name.
The produced program relations include \code{VH}, \code{FH}, \code{HFH}, and
\code{IM}.  We next briefly discuss these relations.

The program relations consumed by the 0CFA analysis contain basic program facts.
For instance, \code{MputStatFldInst} is a relation containing each tuple
($m$,$f$,$v$) such that method $m$ in the input Java program contains a
putstatic instruction of the form ``$f$ = $v$", while \code{statIM} is a
relation containing each tuple ($i$,$m$) such that $m$ is the target method of
invokestatic instruction $i$.

The program relations produced by the 0CFA analysis represent points-to
information and the call graph of the input Java program as computed by the
analysis.  Specifically, relations \code{VH}, \code{FH}, and \code{HFH}
represent points-to information for local variables, static fields, and instance
fields, respectively, while relation \code{IM} represents the call graph,
namely, it contain each tuple ($i$,$m$) such that $m$ is a possible target
method of call site $i$.

Metavariables $m$, $f$, $i$, and $v$ above range over values in so-called
{\it program domains} \code{M}, \code{F}, \code{I}, and \code{V}, respectively.
A program domain is a set of values of a fixed kind in the input Java
program.  For instance, \code{M} is the domain representing the set of all
methods in the input Java program, \code{F} is the domain representing the set
of all fields, \code{I} is the domain representing the set of all call sites in
methods in \code{M}, and \code{V} is the domain representing the set of all
local variables of reference type in methods in \code{M}.  Each of these domains
is produced by a separate analysis with the corresponding name.  Notice
that the analyses producing these domains run upfront because these domains are
consumed by the analyses that produce relations such as \code{MputStatFldInst}
and \code{statIM}, which in turn are consumed by the desired 0CFA analysis.

During execution, Petablox dumps intermediate and final results to files in the
directory specified by property \code{petablox.out.dir}, whose default value is
\code{[petablox.work.dir]/petablox_output/} and typically does not need to be changed
by users.  For the above example, this directory is
\code{<WORK_DIR>/petablox_output/}.

The verbosity of Petablox's output is controlled by property
\code{petablox.verbose}, whose default value is 1.  At verbosity level 0, the above
command produces less voluminous output of the form:

\begin{framed}
{\small
\begin{verbatim}
Buildfile: build.xml

run:
     [java] Petablox run initiated at: Mar 13, 2011 10:35:01 PM
     [java] Petablox run completed at: Mar 13, 2011 10:35:28 PM
     [java] Total time: 00:00:26:297 hh:mm:ss:ms

BUILD SUCCESSFUL
Total time: 26 seconds
\end{verbatim}
}
\end{framed}

\section{Points-to and Call-Graph Analyses}

Petablox provides several analyses with different precision/scalability tradeoff for
computing points-to and call-graph information of Java programs.  All these
analyses are defined under directory \code{src/chord/analyses/alias/}.

In all these analyses, the points-to and call-graph information is computed
simultaneously (called ``on-the-fly call-graph construction" in the literature
in contrast to ``ahead-of-time call-graph construction" in which the call graph
is computed first followed by points-to information).  On-the-fly approaches are
more precise because, in a dynamically dispatching language like Java, as more
points-to facts are discovered, more (dynamically dispatched) methods are deemed
reachable, thereby growing the call graph; the code in these newly added methods
in turn results in more points-to facts.

Broadly, the various analyses are characterized by two aspects (1) flow
sensitive vs. insensitive and (2) context sensitive vs. insensitive.

Flow-insensitive analysis computes a single abstract heap whereas flow-sensitive
analysis computes per-program-point abstract heaps.  Context-insensitive
analysis analyzes each method at most once (i.e. in a single abstract context),
whereas context-sensitive analyses potentially analyze each method multiple
times, in different abstract contexts.  Thus, flow- and context-sensitive
analyses are more precise but less scalable than flow- and context-insensitive
analyses, respectively.

Flow-sensitive analysis does not offer much precision over flow-insensitive
analysis in practice, especially in the absence of strong updates and in the
presence of SSA (Static Single Assignment form), a program representation that
renders a flow-insensitive analysis almost as precise as a flow-sensitive
analysis.  Since analyses in Petablox currently perform only weak updates, and
since they all operate on an SSA form of the input Java program by default, the
rest of this section focuses only on flow-insensitive analysis, which is the
predominant kind of points-to/call-graph analysis in Petablox.

We describe context-insensitive analysis first because understanding the
concepts behind it will help understand the more sophisticated context-sensitive
analyses.  We first recall some relevant program domains:

\texonly{\newpage}

\begin{itemize}
\item M is the domain of all methods.
\item I is the domain of all method call sites.
\item F is the domain of all (instance and static) fields.
\item V is the domain of all local variables of reference type.
\item H is the domain of all object allocation sites.
\end{itemize}

\subsection{Context-Insensitive Analysis}

The context-insensitive points-to/call-graph analysis derives its name from the
fact that it analyzes each method in a single abstract context.
Additionally, the analysis is:
\begin{itemize}
\item
{\it object-insensitive}: it cannot distinguish between
different objects created at the same object allocation site; the reason is that
it treats each object allocation site as a separate abstract memory location.
\item
{\it field-sensitive}: it distinguishes between
different instance fields of the same object.
\item
{\it array-insensitive}: it cannot distinguish between
different elements of the same array; all array elements are modeled using a
distinguished hypothetical instance field (that has index 0 in domain F).
\end{itemize}

The following command runs this analysis on the program specified
by directory \code{<WORK_DIR>} (see Chapter \ref{chap:setup} for how to setup a
program):

\begin{framed}
\begin{verbatim}
ant -Dpetablox.work.dir=<WORK_DIR> -Dpetablox.run.analyses=cipa-0cfa-dlog run
\end{verbatim}
\end{framed}

This analysis outputs the following relations:

\begin{itemize}
\item {\it Call-graph information:}
\begin{itemize}
\item
rootM subset M contains the set of entry methods that may be reachable; this
includes the program's main method as well as each class initializer method
that may be reachable from the main method.
\item
reachableM subset M contains the set of methods that may be reachable from the
program's main method.
\item
IM subset (I $\times$ M) contains tuples \texttt{(i,m)} such that call site
\texttt{i} may call method \texttt{m}.
%\item
% MM subset (M $\times$ M) contains tuples \texttt{(m1,m2)} such that
%method \texttt{m1} may call method \texttt{m2}.
%\item
%reachableI subset I contains the set of call sites that may be reachable.
%\item
%reachableT subset T contains the set of classes that may be reachable.
\end{itemize}
\item
{\it Points-to information:}
\begin{itemize}
\item
FH subset (F $\times$ H) contains tuples \texttt{(f,h)} such that static field
(i.e.  global variable) \texttt{f} may point to an object allocated at site
\texttt{h}.
\item
VH subset (V $\times$ H) contains tuples \texttt{(v,h)} such that local variable
\texttt{v} may point to an object allocated at site \texttt{h}.
\item
HFH subset (H $\times$ F $\times$ H) contains tuples \texttt{(h1,f,h2)} such
that instance field \texttt{f} of some object allocated at site \texttt{h1} may
point to some object allocated at site \texttt{h2}.
\end{itemize}
\end{itemize}

\subsection{Context-Sensitive Analysis}

In a context-sensitive analysis, each method is analyzed in potentially multiple
contexts, in contrast to a context-insensitive analysis, in which each
method is always analyzed in a single context.  Additionally, a
context-sensitive analysis can be {\it object-sensitive} in that it can distinguish
between different objects created at the same object allocation site, whereas
a context-insensitive analysis is always object-insensitive.

Different kinds of context- and object-sensitivity have been proposed in the
literature.  Petablox's context-sensitive points-to/call-graph analysis is highly
parametric in the kind of context- and object-sensitivity.  It allows:
\begin{itemize}
\item
each method to be analyzed using a different kind of context sensitivity,
namely, one of context insensitivity, k-CFA, k-object-sensitivity, and
copy-context-sensitivity;
\item
each local variable to be analyzed context sensitively or insensitively; and
\item
a different 'k' value to be used for each object allocation site and method
call site.
\end{itemize}

The analysis can be called multiple times and in each invocation it can
incorporate feedback from a client to adjust the precision of the points-to
information and call graph computed subsequently by the points-to/call-graph
analyses.  Clients can indicate in each invocation:

\begin{itemize}
\item
Which methods must be analyzed context sensitively (in addition to those already being
analyzed context sensitively in the previous invocation of this analysis) and using
what kind of context sensitivity; the remaining methods will be analyzed context
insensitively (that is, in the lone 'epsilon' context).
\item
Which local variables of reference type must be analyzed context sensitively (in
addition to those already being analyzed context sensitively in the previous
invocation of this analysis); the remaining ones will be analyzed context insensitively
(that is, their points-to information will be tracked in the lone 'epsilon' context).
\item
The object allocation sites and method call sites whose 'k' values must be incremented
(over those used in the previous invocation of this analysis).
\end{itemize}

The following command runs this analysis on the program specified
by directory \code{<WORK_DIR>} (see Chapter \ref{chap:setup} for how to setup a
program):

\begin{framed}
\begin{verbatim}
ant -Dpetablox.work.dir=<WORK_DIR> -Dpetablox.run.analyses=cspa-hybrid-dlog run
\end{verbatim}
\end{framed}

System properties recognized by this analysis are:
\begin{itemize}
\item
\code{petablox.inst.ctxt.kind}: the kind of context sensitivity to use for each
instance method (and all its locals).  It may be 'ci' (context insensitive), 'cs'
(k-CFA), or 'co' (k-object-sensitive).  It is 'ci' by default.
\item
\code{petablox.stat.ctxt.kind}: the kind of context sensitivity to use for each
static method (and all its locals).  It may be 'ci' (context insensitive), 'cs'
(k-CFA), or 'co' (copy-context-sensitive).  It is 'ci' by default.
\item
\code{petablox.ctxt.kind}: the kind of context sensitivity to use for each method
(and all its locals).  It may be one of 'ci', 'cs', or 'co', and serves as
shorthand for properties \code{petablox.inst.ctxt.kind} and \code{petablox.stat.ctxt.kind}.
\item
\code{petablox.kobj.k} and \code{petablox.kcfa.k}: the 'k' value to use for each object
allocation site and each method call site, respectively.
It is 1 by default.
\end{itemize}

The above command runs the most general form of context-sensitive
points-to/call-graph analysis,
allowing different kinds of context- and object-sensitivity to be mixed in the analysis of
the same program, but this flexibility comes at the cost of efficiency.
Hence, Petablox provides three specialized forms of this analysis that more
efficiently handle certain common instantiations of this analysis that do not
arbitrarily mix different kinds of context- and object-sensitivity.  We next describe these
three specialized analyses.

If \code{petablox.ctxt.kind} is 'ci' (the default value),
the above command essentially performs a purely context-insensitive
analysis; in this case, it is preferable to run the following more efficient
implementation of that analysis:

\begin{framed}
\begin{verbatim}
ant -Dpetablox.work.dir=<WORK_DIR> -Dpetablox.run.analyses=cspa-0cfa-dlog run
\end{verbatim}
\end{framed}

If \code{petablox.ctxt.kind} is 'cs' and \code{petablox.kcfa.k} is K,
the above command essentially performs a purely K-CFA analysis; in this case, it
is preferable to run the following more efficient implementation of that
analysis:

\begin{framed}
\begin{verbatim}
ant -Dpetablox.work.dir=<WORK_DIR> -Dpetablox.ctxt.kind=cs -Dpetablox.kcfa.k=K -Dpetablox.run.analyses=cspa-kcfa-dlog run
\end{verbatim}
\end{framed}

If \code{petablox.ctxt.kind} is 'co' and \code{petablox.kobj.k} is K,
the above command essentially performs a purely K-object-sensitive analysis; in
this case, it is preferable to run the following more efficient implementation
of that analysis:

\begin{framed}
\begin{verbatim}
ant -Dpetablox.work.dir=<WORK_DIR> -Dpetablox.ctxt.kind=co -Dpetablox.kobj.k=K -Dpetablox.run.analyses=cspa-kobj-dlog run
\end{verbatim}
\end{framed}

Each of the above four commands outputs the following relations:

\begin{itemize}
\item
{\it Context information:}
\begin{itemize}
\item
C is the domain of all abstract contexts and abstract objects.  Each
element in this domain is a sequence of zero or more sites, where each site can
be a call site or an object allocation site.  A sequence may have mixed call
sites and object allocation sites.  The most significant site (i.e. the first
site) in each sequence is called the head; the remaining sub-sequence is called
the tail.  The below three relations relate a sequence with its head and tail.
\item
CC subset (C $\times$ C) contains tuples \texttt{(c1,c2)} such that context
\texttt{c2} is the tail of context \texttt{c1}.
\item
CH subset (C $\times$ H) contains tuples \texttt{(c,h)} such that object
allocation site \texttt{h} is the head of context \texttt{c}.
\item
CI subset (C $\times$ I) contains tuples \texttt{(c,i)} such that call site
\texttt{i} is the head of context \texttt{c}.
\end{itemize}
\item
{\it Call-graph information:}
\begin{itemize}
\item
rootCM subset (C $\times$ M) contains tuples \texttt{(c,m)} such that method
\texttt{m} is an entry method in context \texttt{c}.
\item
reachableCM subset (C $\times$ M) contains tuples \texttt{(c,m)} such that
method \texttt{m} may be reachable in context \texttt{c}.
\item
CICM subset (C $\times$ I $\times$ C $\times$ M) contains tuples
\texttt{(c1,i,c2,m)} such that call site \texttt{i} in context \texttt{c1} may
call method \texttt{m2} in context \texttt{c2}.
\end{itemize}
\item
{\it Points-to information:}
\begin{itemize}
\item
FC subset (F $\times$ C) contains tuples \texttt{(f,o)} such that static field
(i.e. global variable) \texttt{f} may point to object \texttt{o}.
\item
CVC subset (C $\times$ V $\times$ C) contains tuples \texttt{(c,v,o)} such that
local variable \texttt{v} may point to object \texttt{o} in context \texttt{c}
of that variable's declaring method.  Note that both \texttt{o} and \texttt{c}
are elements of domain C.
\item
CFC subset (C $\times$ F $\times$ C) contains tuples \texttt{(o1,f,o2)} such
that instance field \texttt{f} of object \texttt{o1} may point to object
\texttt{o2}.  \end{itemize}
\end{itemize}

\subsection{Call Graph Variants}

Petablox offers two variants of each of the context-insensitive and
context-sensitive baseline call graphs presented above, called {\it abbreviated
thread-insensitive} and {\it abbreviated thread-sensitive} call graphs.  We
explain these two variants for the context-insensitive case; they are analogous
for the context-sensitive case.  Consider the following example Java program:

\begin{framed}
\begin{verbatim}
public class T extends java.lang.Thread {
    static {
i1:     m1();
    }
    public static void m1() { }
    public static void m2() { }
    public static void m3() { }
    public void run() {
i2:     m2();
    }
    public static void main() {
i3:     m3();
        T t = ...;
i4:     t.start();
    }
}
\end{verbatim}
\end{framed}

The baseline call graph for this program is 
rootM = \{ main, \code{<clinit>} \}, 
reachableM = \{ main, \code{<clinit>}, m1, m2, m3, start, run \}, and
IM = \{ (i1, m1), (i2, m2), (i3, m3), (i4, start), (i5, run) \}, where methods
main, m1, m2, m3, and run are those declared in class C, method \code{<clinit>}
is the class initializer method of class C, and method start denotes the
\code{java.lang.Thread.start()} method which (not shown) contains a call site labeled i5 that calls
\code{this.run()}.

An {\it abbreviated thread-insensitive} call graph excludes from the baseline call
graph all methods that are unreachable from the main method in that call graph.  In other words, it
excludes all class initializer methods as well as methods that are reachable
only from such methods.
For the above example, this call graph is
rootM = \{ main \},
reachableM = \{ main, m2, m3, start, run \}, and
IM = \{ (i2, m2), (i3, m3), (i4, start), (i5, run) \}.  Notice that methods
\code{<clinit>} and m1 are omitted from reachableM, and call edge (i1, m1) is
omitted from IM, since m1 is unreachable from the main method.

An {\it abbreviated thread-sensitive} call graph excludes from the baseline call
graph all methods that are unreachable from the main method, like the
abbreviated thread-insensitive call graph, and additionally treats the
\code{java.lang.Thread.start()} method as a root method and does not include
call edges that target that method.
For the above example, this call graph is
rootM = \{ main, start \},
reachableM = \{ main, m2, m3, start, run \}, and
IM = \{ (i2, m2), (i3, m3), (i5, run) \}.  Notice that the start method is
added to rootM and call edge (i4, start) is omitted from IM.

The above two call graph variants are provided to simplify expressing
certain forms of reachability:
\begin{itemize}
\item
An {\it abbreviated} call graph allows to ignore reachability from
class initializer methods;
this is used, for instance, in Petablox's static datarace analysis to
suppress reporting races on statements reachable only from class initializer
methods, which are typically false races in practice.
\item
A {\it thread-sensitive} call graph additionally allows to ignore
``inter-thread" reachability and focus on ``intra-thread'' reachability; this is
used, for instance, in Petablox's static datarace analysis to prevent reporting
false races (t1,e1,t2,e2) where statement e1 (or e2 resp.) is unreachable from
abstract thread t1 (or t2 resp.).
\end{itemize}

The following commands can be used to produce the above two call graph variants
for the context-insensitive and context-sensitive cases, respectively, for the
program specified by directory \code{<WORK_DIR>} (see Chapter \ref{chap:setup}
for how to setup a program):

\begin{framed}
\begin{verbatim}
ant -Dpetablox.work.dir=<WORK_DIR> -Dpetablox.run.analyses=thrOblAbbrCICG-dlog run

ant -Dpetablox.work.dir=<WORK_DIR> -Dpetablox.run.analyses=thrSenAbbrCICG-dlog run

ant -Dpetablox.work.dir=<WORK_DIR> -Dpetablox.run.analyses=thrOblAbbrCSCG-dlog run

ant -Dpetablox.work.dir=<WORK_DIR> -Dpetablox.run.analyses=thrSenAbbrCSCG-dlog run
\end{verbatim}
\end{framed}


\section{Static Datarace Analysis}

The following command runs Petablox's static datarace analysis on the
program specified by directory \code{<WORK_DIR>} (see Chapter \ref{chap:setup}
for how to setup a program):

\begin{framed}
\begin{verbatim}
ant -Dpetablox.work.dir=<WORK_DIR> -Dpetablox.run.analyses=datarace-java run
\end{verbatim}
\end{framed}

Directory \code{examples/datarace_test/} provides a toy Java program on which to
run the datarace analysis.  First run {\tt ant} in that directory (in order to
compile the program's {\tt .java} files to {\tt .class} files) and then run the
above command with \code{<WORK_DIR>} replaced by \code{examples/datarace_test/}.
Upon successful completion, the following files should be produced in directory
\code{examples/datarace_test/chord_output/}:

\begin{itemize}
\item
File \code{dataraces_by_fld.html}, listing all dataraces grouped by the field on
which they occur; all dataraces on the same instance field or the same static
field are listed in the same group, and so are all dataraces on array elements.
\item
File \code{dataraces_by_obj.html}, listing all dataraces grouped by the abstract
object on whose field they occur; dataraces on all static fields are listed in
the same group, and so are dataraces on different instance fields of the same
abstract object.
\end{itemize}

\section{Static Deadlock Analysis}

The following command runs Petablox's static deadlock analysis on the
program specified by directory \code{<WORK_DIR>} (see Chapter \ref{chap:setup}
for how to setup a program):

\begin{framed}
\begin{verbatim}
ant -Dpetablox.work.dir=<WORK_DIR> -Dpetablox.run.analyses=deadlock-java run
\end{verbatim}
\end{framed}

Directory \code{examples/deadlock_test/} provides a toy Java program on which to
run the deadlock analysis.  First run {\tt ant} in that directory (in order to
compile the program's {\tt .java} files to {\tt .class} files) and then run the
above command with \code{<WORK_DIR>} replaced by \code{examples/deadlock_test/}.
Upon successful completion, the file \code{deadlocks.html} should be produced in
directory \code{examples/deadlock_test/petablox_output/}.

