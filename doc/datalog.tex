\xname{datalog}
\chapter{Datalog Analysis}
\label{chap:datalog}

A common way to rapidly prototype an analysis in Petablox is using a declarative
logic-programming language called Datalog.  This chapter describes all aspects
of Datalog analyses in Petablox.  Section \ref{sec:writing-datalog} explains how to
write and run a Datalog analysis while Section \ref{sec:tuning-datalog} explains how
to tune its performance.

\section{Writing a Datalog Analysis}
\label{sec:writing-datalog}

A Datalog analysis declares a bunch of input/output program relations, each over
one or more program domains, and provides a bunch of rules (constraints)
specifying how to compute the output relations from the input relations.  It can
be defined in any file with suffix {\tt .dlog} in any directory in the path
specified by property \code{petablox.dlog.analysis.path}.  An example Datalog
analysis is shown below:

\texonly{\newpage}

\begin{framed}
{\small
\begin{verbatim}
    # name=datarace-dlog

    # Program domains
    .include "E.dom"
    .include "F.dom"
    .include "T.dom"

    # BDD variable order
    .bddvarorder E0xE1_T0_T1_F0

    # Input/intermediate/output program relations
    field(e:E0,f:F0) input
    write(e:E0) input
    reach(t:T0,e:E0) input
    alias(e1:E0,e2:E1) input
    escape(e:E0) input
    unguarded(t1:T0,e1:E0,t2:T1,e2:E1) input
    hasWrite(e1:E0,e2:E1)
    candidate(e1:E0,e2:E1) 
    datarace(t1:T0,e1:E0,t2:T1,e2:E1) output

    # Analysis constraints
    hasWrite(e1,e2) :- write(e1).
    hasWrite(e1,e2) :- write(e2).
    candidate(e1,e2) :- field(e1,f), field(e2,f), hasWrite(e1,e2), e1 <= e2.
    datarace(t1,e1,t2,e2) :- candidate(e1,e2), reach(t1,e1), reach(t2,e2), \
        alias(e1,e2), escape(e1), escape(e2), unguarded(t1,e1,t2,e2).
\end{verbatim}
}
\end{framed}

Any line that begins with a ``{\tt \#}" is regarded a comment, except a
line of the form ``\code{# name=<ANALYSIS_NAME>}'', which specifies the name
\code{<ANALYSIS_NAME>} of the analysis.
Each such analysis is expected to have exactly one such line.
The above analysis is named {\tt datarace-dlog}.

The ``\code{.include "<DOM_NAME>.dom"}'' lines specify each domain named
\code{<DOM_NAME>} that is needed by the analysis, i.e., each domain over which
any relation that is input/output by the analysis is defined. 
The declaration of each such relation specifies the relation's schema:
the list of domains over which the relation is defined.
If the same domain appears multiple times in a relation's schema then
distinct non-negative integers must be used to distinguish them; for instance,
in the above example, {\tt candidate} is a binary relation, both of whose
attributes have domain {\tt E}, and they are distinguished as {\tt E0} and {\tt E1}.

Each relation is represented symbolically (as opposed to explicitly)
using a graph-based data structure called a Binary Decision Diagram (BDD for short).
Each domain containing N elements is assigned log2(N) BDD variables.
The size of a BDD and the efficiency of operations that are performed on it when the above
analysis runs depends heavily
on the order of these BDD variables.
The ``\code{.bddvarorder <BDD_VAR_ORDER>}'' line enables the 
analysis writer to specify this order.
It must list all domains along with their numerical suffixes, separated
by `{\tt \_}' or `{\tt x}'.
Using a `{\tt \_}' between two domains, such as {\tt T0\_T1}, means that the BDD variables assigned
to domain {\tt T0} precede those assigned to domain {\tt T1} in the BDD variable
order for this analysis.
Using a `{\tt x}' between two domains, such as {\tt E0xE1}, means that the
BDD variables assigned to domains {\tt E0} and {\tt E1}
will be interleaved in the BDD variable order for this analysis.
See Section \ref{sec:bdd} for more details on BDD representations.

Each rule in the analysis is a Horn clause of the form
``{\tt R(t) :- R1(t1), ..., Rn(tn)}"
meaning that if relations {\tt R1}, ..., {\tt Rn} contain tuples {\tt t1}, ..., {\tt tn}
respectively, then relation {\tt R} contains tuple {\tt t}.
A backslash may be used at the end of a line to break long rules for readability.
The BDD-based Datalog solver bddbddb used in Petablox does not apply any
sophisticated optimizations to simplify the rules; besides the BDD variable order,
the manner in which these rules are expressed heavily affects the performance of
the solver.  For instance, an important manual optimization involves breaking down
long rules into multiple shorter rules communicating via intermediate relations.
See Section \ref{sec:tuning-datalog} for hints on tuning the performance
of Datalog analyses.

Running a Datalog analysis is no different from running any other kind of
analysis in Petablox.  See Section \ref{sec:running-predefined} for an example of
running a Datalog analysis predefined in Petablox.  Also, see Chapter
\ref{chap:running} for how to run an analysis in general, whether written in
Java or in Datalog, and whether predefined or user-defined.

Each program domain used in a Datalog analysis must be
produced by an analysis written imperatively in Java (see Section
\ref{sec:program-dom}).
Likewise, each program relation declared as an input relation in a Datalog analysis must be
produced either by another Datalog analysis (i.e., declared as an output relation
of that analysis) or by an analysis written imperatively in Java (see Section
\ref{sec:program-rel}).
Running a Datalog analysis will result in a runtime error if any of the program
domains used in the analysis or any of the program relations declared as input
relations in the analysis are not produced by any other analysis.

\section{Tuning Performance}
\label{sec:tuning-datalog}

There are several tricks analysis writers can try to improve the
performance of bddbddb, the Datalog solver used by Petablox, often
by several orders of magnitude.
Try these tricks by running the following command:

\begin{framed}
\begin{verbatim}
ant -Ddlog.file=<FILE> -Dwork.dir=<DIR> solve
\end{verbatim}
\end{framed}

where {\tt <FILE>} is the file defining the Datalog analysis
to be tuned, and {\tt <DIR>} is the directory containing the
program domains ({\tt *.dom} files) and program relations ({\tt *.bdd} files)
consumed by the analysis (this is by default the
\code{petablox_output/bddbddb/} directory generated
by a previous run of Petablox.

\begin{enumerate}

\item

Set properties \verb+noisy=yes+, \verb+tracesolve=yes+, and \verb+fulltracesolve=yes+
on the above command line and observe which rule gets ``stuck" (i.e., takes several seconds to solve).
\verb+fulltracesolve+ is seldom useful, but \verb+noisy+ and \verb+tracesolve+ are
often very useful.  Once you identify the rule that is getting stuck, it
will also tell you which relations and which domains used in that rule,
and which operation on them, is taking a long time to solve.  Then try
to fix the problem with that rule by doing either or both of the following:
\begin{itemize}
\item
Break down the rule into multiple rules by creating intermediate relations (the more
relations you have on the RHS of a rule the slower it generally takes to solve
that rule).
\item
Change the relative order of the domains of those
relations in the BDD variable order
(note that you can use either `\_' or `x' between a pair of domains).
\end{itemize}

\item

Once you have ensured that none of the rules is getting ``stuck",
you will notice that some rules are applied too many times, and so
although each application of the rule itself isn't taking too much
time, the cumulative time for the rule is too much.  After finishing
solving a Datalog analysis, bddbddb prints how long each rule took to
solve (both in terms of the number of times it was applied and the
cumulative time it took).  It sorts the rules in the order of the
cumulative time.  You need to focus on the rules that took the most
time to solve (they will be at the bottom of the list).  Assuming you
removed the problem of rules getting ``stuck", the rules will roughly
be in the order of the number of times they were applied.  Here is an
example:

\begin{framed}
\begin{verbatim}
OUT> Rule VH(u:V0,h:H0) :- VV(u:V0,v:V1), VH(v:V1,h:H0), VHfilter
(u:V0,h:H0).
OUT>    Updates: 2871
OUT>    Time: 6798 ms
OUT>    Longest Iteration: 0 (0 ms)
OUT> Rule IM(i:I0,m:M0) :- reachableI(i:I0), specIMV(i:I0,m:M0,v:V0), VH(v:V0,_:H0).
OUT>    Updates: 5031
OUT>    Time: 6972 ms
OUT>    Longest Iteration: 0 (0 ms)
\end{verbatim}
\end{framed}

Notice that the second rule was applied 5031 times whereas the first
was applied 2871 times.  More importantly, the second rule took 6972
milliseconds in all, compared to 6798 for the first rule.  Hence, you
should focus on the second rule first, and try to speed it up.  This
means that you should focus only on relations IM, reachableI, specIMV,
and VH, and the domains I0, M0, V0, and H0.  Any changes you make that
do not affect these relations and domains are unlikely to make your
solving faster.  In general, look at the last few rules, not just the
last one, and try to identify the ``sub-analysis" of the Datalog analysis
that seems problematic, and then focus on speeding up just that sub-
analysis.

\item

You can add the \verb+.split+ keyword at the end of certain rules as a
hint to bddbddb to decompose those rules into simpler ones that can be
solved faster.  You can also set property \verb+split_all_rules=yes+ as shorthand
for splitting all rules without adding the \verb+.split+ keyword to any of
them, though I seldom find splitting all rules helpful.

\item

You can try to decompose a single Datalog analysis file into two separate Datalog analysis
files.  Of course, you cannot separate mutually-recursive rules into two
different analyses, but if you unnecessarily club
together rules that could have gone into different analyses, then they
can put conflicting demands on bddbddb (e.g., on the BDD variable order).
So if rule 2 uses the result of rule 1 and rule 1 does not use the result of
rule 2, then put rule 1 and rule 2 in separate Datalog analyses.

\item

Observe the sizes of the BDDs representing the relations that are
input and output.  bddbddb prints both the number of tuples in each
relation and the number of nodes in the BDD.  Try changing the
BDD variable order for the domains of the relation, and observe how the
number of nodes in the BDD for that relation change.  You will
notice that some orders perform remarkably better than others.  Then
note down these orders as invariants that you will not violate as
you tweak other things.

\item

The relative order of values *within* domains (e.g.,
in domains named \verb+M+, \verb+H+, \verb+C+, etc. in Petablox) affects the
performance of bddbddb, but
I've never tried changing this and studying its effect.  It might be
worth trying.  For instance, John Whaley's PLDI'04 paper describes a
specific way in which he numbers contexts (in domain \verb+C+) and that it was
fundamental to the speedup of his ``infinity"-CFA points-to analysis.

\item

Finally, it is worth emphasizing that BDDs are not magic.
If your algorithm itself is fundamentally hard to scale, then BDDs are
unlikely to help you a whole lot.  Secondly, many things are awkward to
encode as integers (e.g., the abstract contexts in domain \verb+C+ 
in Petablox) or as Datalog rules.
For instance, I've noticed that summary-based context-sensitive program
analyses are hard to express in Datalog.  The may-happen-in-parallel
analysis provided in Petablox shows a relatively simple kind of summary-based
analysis that uses the Reps-Horwitz-Sagiv tabulation algorithm.  But this
is as far as I could get---more complicated summary-based algorithms are
best written in Java itself instead of Datalog.
\end{enumerate}

