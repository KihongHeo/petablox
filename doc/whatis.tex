\xname{whatis}
\chapter{What is Petablox?}
\label{chap:whatis}

Petablox is a program analysis platform that enables users to productively design,
implement, evaluate, and combine a broad variety of static and dynamic program
analyses for Java bytecode. It has the following key features:

\begin{itemize}
\item

It provides various off-the-shelf analyses (e.g., various may-alias and
call-graph analyses; thread-escape analysis; static slicing analysis; static and
dynamic concurrency analyses for finding races, deadlocks, and atomicity
violations; etc.)

\item

It allows users to express a broad range of analyses, including both static and
dynamic analyses, analyses written imperatively in Java or declaratively in
Datalog, summary-based as well as cloning-based inter-procedural
context-sensitive analyses, iterative refinement-based analyses, client-driven
analyses, and combined static/dynamic analyses.

\item

It executes analyses in a demand-driven fashion, caches results computed by each
analysis for reuse by other analyses without re-computation, and can execute
analyses without dependencies between them in parallel.

\item

It guarantees that the result is the same regardless of the order in which
different analyses are executed; moreover, results can be shared across
different runs.
\end{itemize}

Petablox is intended to work on a variety of platforms, including Linux,
Windows/Cygwin, and MacOS.  It is open-source software distributed under
the \bsdlicense.  Improvements by users are welcome and encouraged.  The project
website is \petabloxweb.
