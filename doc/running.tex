\xname{running}
\chapter{Running an Analysis}
\label{chap:running}

This chapter describes how to run an analysis in Petablox.  Broadly, there are two
kinds of analyses in Petablox: those written imperatively in Java and those written
declaratively in Datalog. They are summarized in the following table:

\begin{mytable}{|l||c|c|}
\hline
Kind:
	& imperative (see Chapter \ref{chap:writing})
	& declarative (see Chapter \ref{chap:datalog}) \\
\hline
Location:
	& \begin{tabular}{c}
	  a {\tt .class} file in the path denoted by property \\
      \code{petablox.java.analysis.path} compiled \\
      from a \code{@Petablox}-annotated class implementing \\
      interface \javadoc{petablox.project.ITask}{petablox/project/ITask.html}
      \end{tabular}
	& \begin{tabular}{c}
      a {\tt .dlog} file in the path denoted by property \\
      \code{petablox.dlog.analysis.path}
      \end{tabular} \\
\hline
Name:
	& via stmt \verb+name="<NAME>"+ in {\tt @Petablox} annotation
	& via line ``\verb+# name=<NAME>+'' in {\tt .dlog} file \T \\
\hline
\end{mytable}

In its most general form, the command for running an analysis is as follows:

\begin{framed}
\begin{verbatim}
ant -Dpetablox.work.dir=<WORK_DIR> -Dpetablox.run.analyses=<ANALYSIS_NAME> \
    -Dpetablox.dlog.analysis.path=<DLOG_ANALYSIS_PATH> \
    -Dpetablox.java.analysis.path=<JAVA_ANALYSIS_PATH> run
\end{verbatim}
\end{framed}

where:
\begin{itemize}
\item
\code{<WORK_DIR>} is a directory containing a file named
\code{petablox.properties} that defines various properties of the Java program to be
analyzed that might be needed by the analysis being run, such as the program's
main class, the program's application classpath, etc.
(see Chapter \ref{chap:setup}).
\item
\code{<ANALYSIS_NAME>} is the name of the analysis to run.  More generally, it
can be a comma-separated list of names of analyses to run in order.
\item
\code{<JAVA_ANALYSIS_PATH>} is a path specifying all imperative analyses written
in Java that might be needed to run the desired analysis.
\item
\code{<DLOG_ANALYSIS_PATH>} is a path specifying all declarative analyses
written in Datalog that might be needed to run the desired analysis.
\end{itemize}

Each analysis in Petablox is written modularly, independent of other analyses,
along with lightweight annotations specifying the inputs and outputs of the
analysis.  Petablox's runtime automatically computes dependencies between analyses
(e.g., determines which analysis produces as output a result that is needed as
input by another analysis).  Before running the desired analysis named
\code{<ANALYSIS_NAME>}, Petablox recursively runs other analyses until the inputs
to the desired analysis have been computed; it finally runs the desired analysis
to produce the outputs of that analysis.  Chapter \ref{chap:project} explains
this process in detail.  Each of these analyses must occur in the path denoted
by \code{<JAVA_ANALYSIS_PATH>} or \code{<DLOG_ANALYSIS_PATH>}, depending upon
whether the analysis is written imperatively in Java or declaratively in
Datalog, respectively.

Petablox provides shorthand ways for specifying analysis paths by means of the
following six properties:

\begin{mytable}{|c|l|l|l|}
\hline
Analysis Kind
	& \multicolumn{1}{c|}{Predefined}
	& \multicolumn{1}{c|}{User-defined}
	& \multicolumn{1}{c|}{All} \\
\hline
imperative
	& \code{petablox.std.java.analysis.path}
	& \code{petablox.ext.java.analysis.path}
	& \code{petablox.java.analysis.path} \\
\hline
declarative
	& \code{petablox.std.dlog.analysis.path}
	& \code{petablox.ext.dlog.analysis.path}
	& \code{petablox.dlog.analysis.path}
\T \\
\hline
\end{mytable}

The paths specified by the \code{petablox.std.*.analysis.path} properties
conventionally include all ``standard'' analyses: analyses that are predefined
in Petablox.  The default value of each of these properties is the absolute path
of file \code{petablox.jar}.  Normally, users must not change the value of these
properties.

The paths specified by the \code{petablox.ext.*.analysis.path} properties
conventionally include all ``external'' analyses: analyses that are written by
users.  The default value of each of these properties is the empty path.

The paths specified by the \code{petablox.*.analysis.path} properties include
{\it all} analyses: both standard and external ones.  The default value of each
of these properties is simply the concatenation of the values of the
corresponding two properties above that specify the paths of standard and
external analyses.  Normally, users must not change the value of these
properties.

Running the above command can cause Petablox to report a runtime error in the
following scenarios:

\begin{itemize}
\item
Either no included analysis or multiple included analyses are named
\code{<ANALYSIS_NAME>}.
\item
A result declared as consumed by the analysis named \code{<ANALYSIS_NAME>}
(or by some analysis on which the specified analysis is dependent directly or
transitively) is either not declared as produced by any included analysis or
is declared as produced by multiple included analyses.
\end{itemize}

To fix the error resulting from the ``missing analysis'' case in both the above
scenarios, simply include the missing analysis in the path specified by
properties \code{petablox.*.analysis.path}.

To fix the error resulting from the ``ambiguous analysis'' case in both the
above scenarios, the names {\tt A1}, ..., {\tt An} of all analyses that were
involved in the ambiguity are provided in the error report.  Suppose {\tt Ai}
is the desired analysis from this set.  Then, one way to fix the error is to
exclude all analyses in that set except analysis {\tt Ai} from the path
specified by properties \code{petablox.*.analysis.path}.  A better way is to
specify the names of {\it multiple} analyses in the value of property
\code{petablox.run.analyses} (recall that this property allows a comma-separated
list of names of analyses to be run in order).  Specifically, the value of
this property must specify the name of the desired analysis {\tt Ai}
{\it before} the name of the analysis that caused the ambiguity error.

The above command is for users who have defined their own analyses and wish
to run them.  The following simpler command that uses the default values for
properties \code{petablox.*.analysis.path} suffices for users who only wish to run
analyses predefined in Petablox:

\begin{framed}
\begin{verbatim}
ant -Dpetablox.work.dir=<WORK_DIR> -Dpetablox.run.analyses=<ANALYSIS_NAME> run
\end{verbatim}
\end{framed}

Section \ref{sec:running-predefined} illustrates this command using an example
predefined analysis in Petablox.

