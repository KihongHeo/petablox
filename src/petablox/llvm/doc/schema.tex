%\documentclass[letterpaper, 10 pt, conference]{ieeeconf}  
\documentclass[letterpaper, 10 pt, conference]{article}  
%\IEEEoverridecommandlockouts{}
%\overrideIEEEmargins{}

\usepackage{acronym}
\usepackage{url}
\usepackage{graphicx}
\usepackage{listings}
\usepackage{amsmath}
\usepackage{textcomp}
\usepackage{caption}
\captionsetup[lstlisting]{}

% Define any acronyms here
%\acrodef{EA}[EA]{Example Acronym}

% Reference commands
\newcommand{\reflist}[1]{Listing~\ref{#1}}
\newcommand{\reffig}[1]{Figure~\ref{#1}}
\newcommand{\refsec}[1]{Section~\ref{#1}}
\newcommand{\reftable}[1]{Table~\ref{#1}}

\newcommand\tab[1][1cm]{\hspace*{#1}}

% Create style for listings
\lstdefinestyle{example}{%
    %basicstyle=\footnotesize,
    basicstyle=\small,
    breakatwhitespace=false,         
    breaklines=true,                 
    captionpos=b,                    
    frame=single,
    keepspaces=true,                 
    %numbers=left,                    
    numbersep=5pt,                  
    showspaces=false,                
    showstringspaces=false,
    showtabs=false,                  
    tabsize=2,
    escapeinside={@/}{/@}
}

\title{\LARGE \bf
    Relational Schema for LLVM IR
}

\author{Leo St.\ Amour\\% <-this % stops a space
}

\begin{document}

\maketitle
\thispagestyle{empty}
\pagestyle{empty}

\begin{abstract}
    This paper describes the relational schema for LLVM IR.
    Credit goes to cclyzer \url{https://github.com/plast-lab/cclyzer/}.
\end{abstract}

\section{Introduction}
This document will outline the relational schema that we will use to represent the LLVM IR in datalog.
The rules are based upon the relational schema present in the cclyzer project.

\section{Types} % based on type.logic
Type ID:\\
\tab \lstinline{type(id).}
\\
Size:\\
\tab \lstinline{type_size(id, size).}
\\
Unpadded size:\\
\tab \lstinline{type_unpaded_size(id, size).}

\subsection{Primitive Types} % based on primitive-type.logic
Primitive type:\\
\tab \lstinline{primitive_type(id).}
\\
Integer type:\\
\tab \lstinline{intger_type(id).}
\\
Floating point type:\\
\tab \lstinline{fp_type(id).}
\\
Void type:\\
\tab \lstinline{void_type(id).}
\\
Label type:\\
\tab \lstinline{label_type(id).}
% TODO: Metadata type
%\\
%Metadata type:\\
%\tab \lstinline{metadata_type(id).}
% TODO: Selectors (i.e. i1, i8, i32...)

\subsection{Derived Types} % based on derived-type.logic
Derived type:\\
\tab \lstinline{derived_type(id).}
\\
% Function type
Function type:\\
\tab \lstinline{function_type(id).}
\\
Function variable arguments:\\
\tab \lstinline{function_type_varargs(id).}
\\
Function return type:\\
\tab \lstinline{function_type_return(id, type).}
\\
Function parameter:\\
\tab \lstinline{function_type_param(id, index, type).}
\\
Number of function parameters:\\
\tab \lstinline{function_type_nparams(id, n).}
\\
% Pointer type
Pointer type:\\
\tab \lstinline{pointer_type(id).}
\\
Pointer component:\\
\tab \lstinline{pointer_type_component(id, type).}
\\
Pointer address:\\
\tab \lstinline{pointer_type_addr_space(id, addr).}
\\
Check if pointer is first class:\\
%\tab \lstinline{pointer_type_first_class(id, type) :- pointer_type_component(id, type), type_first_class(type).}
\tab \texttt{pointer\_type\_first\_class(id, type) :- \\
            \tab\tab pointer\_type\_component(id, type), \\
            \tab\tab type\_first\_class(type).}
\\
% TODO: rules for each type of pointer?
% Vector type
Vector type:\\
\tab \lstinline{vector_type(id).}
\\
Vector component:\\
\tab \lstinline{vector_type_component(id, type).}
\\
Vector size:\\
\tab \lstinline{vector_type_size(id, size).}
\\
% TODO: rules for each type of vector?
% Aggregate type
Aggregate type:\\
\tab \lstinline{aggregate_type(id).}
\\
% Array types
Array type:\\
\tab \lstinline{array_type(id).}
\\
Array component:\\
\tab \lstinline{array_type_component(id, type).}
\\
Array size:\\
\tab \lstinline{array_type_size(id, size).}
\\
% Struct type 
Struct type:\\
\tab \lstinline{struct_type(id).}
\\
Struct field:\\
\tab \lstinline{struct_type_field(id, index, field).}
\\
Number of struct field:\\
\tab \lstinline{struct_type_nfields(id, n).}
\\
Struct field byte offset:\\
\tab \lstinline{struct_type_field_offset(id, index, offset).}
\\
Struct field bit offset:\\
\tab \lstinline{struct_type_field_bit_offset(id, index, offset).}
% TODO: struct debug information
\\
Opaque type:\\
\tab \lstinline{opaque_type(id).}
\\
First class types:
\tab \lstinline{type_first_class(id) :- type(id), !function_type(id), !void_type(id).}

\section{Functions} % based on function.logic
\noindent Function ID:
\texttt{function(id).}

\noindent Function name:
\texttt{function\_name(id, name).}

\noindent Function return type:
\texttt{function\_type(id, type).}

\noindent Function signature:
\texttt{function\_signature(id, sig).}
% TODO: constant/variable/operand\_in\_function

\noindent Linkage type:
\texttt{function\_linkage\_type(id, link).}

\noindent Visibility:
\texttt{function\_visibility(id, vis).}

\noindent Calling convention:
\texttt{function\_calling\_convention(id, conv).}

\noindent Unnamed address:
\texttt{function\_unnamed\_addr(id).}

\noindent Alignment:
\texttt{function\_alignment(id, align).}

\noindent Garbage collector:
\texttt{function\_gc(id, gc\_name).}

\noindent Personality function:
\texttt{function\_pers\_fn(id, pers\_fn).}

\noindent Attribute:
\texttt{function\_attribute(id, attr).}

\noindent Section (Definitions only):
\texttt{function\_section(id, sect).}

\noindent Number of parameters:
\texttt{function\_nparams(id, n).}

% TODO: second rule for nparams
\noindent Parameters:
\texttt{function\_param(id, index, param).}

\noindent Return parameter attribute:
\texttt{function\_return\_attribute(id, attr).}

\noindent Parameter attribute:
\texttt{function\_param\_attribute(id, index, attr).}

% TODO: param\_by\_value
%\\
%Parameter is passed by value:\\
%\tab \texttt{function\_param\_by\_value(id, index, attr).}

\section{Instructions}
\subsection{Terminator instructions}
The next set of relations represent the terminator instructions.
These are the instructions that end basic blocks.
They produce control flow.

\subsubsection{Return instruction}
Instruction ID:
\texttt{ret\_instruction(id).}

\noindent Terminator:
\texttt{terminator\_instruction(id) :- ret\_instruction(id).}

\noindent Return Value:
\texttt{ret\_instruction\_value(id, value).}

\noindent Void return:
\texttt{ret\_instruction\_void(id).}

% TODO: well formed rules
%\subsubsection{Well formed function rules}
%Well formed:
%\texttt{function\_wellformed(Id) :- function(Id), !function\_illformed(Id).}

\subsubsection{Branch instruction}
Instruction ID:
\texttt{br\_instruction(id).}

\noindent Terminator Instruction:
\texttt{terminator\_instruction(id).}

%\subsubsection{Conditional Branch}
\paragraph{Conditional Branch}
~\\Conditional branch instruction ID:
\texttt{br\_cond\_instruction(id).}

\noindent Condition:
\texttt{br\_cond\_instruction\_condition(id, cond).}

\noindent True case:
\texttt{br\_cond\_instruction\_iftrue(id, label).}

\noindent False case:
\texttt{br\_cond\_instruction\_iffalse(id, label).}

%\subsubsection{Unconditional Branch}
\paragraph{Unconditional Branch}
~\\\noindent Unconditional branch instruction ID:
\texttt{br\_uncond\_instruction(id).}

\noindent Destination:
\texttt{br\_cond\_instruction\_dest(id, label).}

\subsubsection{Unreachable instruction}
Instruction ID:
\texttt{unreachable\_instruction(id).}

\noindent If the instruction is unreachable, it is a terminator:\\
\tab \texttt{terminator\_instruction(Id) :- unreachable\_instruction(Id).}

\subsection{Binary Operation Instructions} 
The binary operations are: \texttt{add, fadd, sub, fsub, mul, fmul, udiv, sdiv, fdiv, urem, srem} and \texttt{frem}.
The bitwise binary operations are: \texttt{shl, lshr, ashr, and, or} and \texttt{xor}.
For an instruction, \texttt{name} from these operations, the relation is:\\

\noindent Instruction ID:
\texttt{name\_instruction(id).}

\noindent First operand:
\texttt{name\_instruction\_first\_operand(id, left).}

\noindent Second operand:
\texttt{name\_instruction\_second\_operand(id, right).}


\subsection{Vector Operation Instructions}
This set of relations represents instructions to insert, access and manipulate vectors.

\subsubsection{Extract element instruction}

\subsubsection{Insert element instruction}

\subsubsection{Shuffle vector instruction}

\subsection{Aggregate Operation Instructions}
This set of relations represents instructions for working with aggregate values.

\subsubsection{Extract value instruction}

\subsubsection{Insert value instruction}


\subsection{Memory Access and Addressing Operations}
The following set of relations represent instructions that read, write and allocate memory.

%%% Alloca instructions

\subsubsection{Allocate instruction} 
\noindent Instruction ID:
\texttt{alloca\_instruction(id).}

\noindent Alignment:
\texttt{alloca\_instruction\_alignment(id, align).}

\noindent Size:
\texttt{alloca\_instruction\_size(id, size).}

\noindent Type:
\texttt{alloca\_instruction\_type(id, type).}

 
%%% Load instructions

\subsubsection{Load instruction}
Instruction ID:
\texttt{load\_instruction(id).}

\noindent Alignment:
\texttt{load\_instruction\_alignment(id, align).}

\noindent Ordering:
\texttt{load\_instruction\_ordering(id, ord).}

\noindent Volatile:
\texttt{load\_instruction\_volatile(id).}

\noindent Atomic:
\texttt{load\_instruction\_atomic(Id) :- \\
        \tab store\_instruction\_ordering(Id, \_).}

\noindent Address:
\texttt{load\_instruction\_address(id, addr).}

%%% Store instructions 

\subsubsection{Store instruction}
Instruction ID:
\texttt{store\_instruction(id).}

\noindent Alignment:
\texttt{store\_instruction\_alignment(id, align).}

\noindent Ordering:
\texttt{store\_instruction\_ordering(id, ord).}

\noindent Volatile:
\texttt{store\_instruction\_volatile(id).}

\noindent Atomic:
\texttt{store\_instruction\_atomic(Id) :- \\
        \tab store\_instruction\_ordering(Id, \_).}

\noindent Value:
\texttt{store\_instruction\_value(id, value).}

\noindent Address:
\texttt{store\_instruction\_address(id, addr).}

%%% Fence instructions

\subsubsection{Fence instruction}
Instruction ID:
\texttt{fence\_instruction(id).}

\noindent Ordering:
\texttt{fence\_instruction\_ordering(id, order).}

%%% CmpXchg instructions

\subsubsection{Compare and exchange instruction}
Instruction ID:
\texttt{cmpxchg\_instruction(id).}

\noindent Ordering:
\texttt{cmpxchg\_instruction\_ordering(id, order).}

\noindent Volatility:
\texttt{cmpxchg\_instruction\_volatile(id).}

\noindent Address:
\texttt{cmpxchg\_instruction\_address(id, addr).}

\noindent Comparison:
\texttt{cmpxchg\_instruction\_cmp(id, cmp).}

\noindent New value:
\texttt{cmpxchg\_instruction\_new(id, value).}

%%% AtomicRMW instructions

\subsubsection{Atomic RMW instruction}
Instruction ID:
\texttt{atomicrmw\_instruction(id).}

\noindent Ordering:
\texttt{atomicrmw\_instruction\_ordering(id, order).}

\noindent Volatility:
\texttt{cmpxchg\_instruction\_volatile(id).}

\noindent Operation:
\texttt{atomicrmw\_instruction\_operation(id, op).}

\noindent Address:
\texttt{atomicrmw\_instruction\_address(id, addr).}

\noindent Value:
\texttt{atomicrmw\_instruction\_value(id, value).}


\subsection{Conversion Instructions}
These relations represent type conversions in LLVM.
The conversion operations are: \texttt{trunc, zext, sext, fptrunc, fpext, fptoui, fptosi, uitofp,
sitofp, ptrtoint, inttoptr, bitcast} and \texttt{addrspacecast}.
For an instruction, \texttt{name} from these operations, the relation is:\\

\noindent Instruction ID:
\texttt{name\_instruction(id).}

\noindent From value:
\texttt{name\_instruction\_from(id, value).}

\noindent To type:
\texttt{name\_instruction\_to\_type(id, type).}

\noindent From type:
\texttt{name\_instruction\_from\_type(id, type).} 


\subsection{Other instructions}
These instructions are ``miscellaneous'' and do not fit in another category.

%%% icmp instruction

\subsubsection{Integer compare instruction}
Instruction ID:
\texttt{icmp\_instruction(id).}

\noindent Condition:
\texttt{icmp\_instruction\_condition(id, predicate)}

\noindent First operand:
\texttt{icmp\_instruction\_first\_operand(id, operand)}

\noindent Second operand:
\texttt{icmp\_instruction\_second\_operand(id, operand)}

%%% fpcmp instruction

\subsubsection{Floating point compare instruction}
Instruction ID:
\texttt{fcmp\_instruction(id).}

\noindent Condition:
\texttt{fcmp\_instruction\_condition(id, predicate)}

\noindent First operand:
\texttt{fcmp\_instruction\_first\_operand(id, operand)}

\noindent Second operand:
\texttt{fcmp\_instruction\_second\_operand(id, operand)}

%%% phi instruction

\subsubsection{Phi instruction}

%%% select instruction

\subsubsection{Select instruction}

%%% call instruction

\subsubsection{Call expression}

%%% va_arg instruction

\subsubsection{Variable argument instruction}
Instruction ID:
\texttt{va\_arg\_instruction(id).}

\noindent Argument list:
% based on va\_arg\_instrction:va\_list[Insn]
\texttt{va\_arg\_instruction\_va\_list(id, ptr).}

\noindent Argument type:
\texttt{va\_arg\_instruction\_type(id, type).}

%%% landingpad instruction

\subsubsection{Landing pad instruction}

%%% catchpad instruction

\subsubsection{Catch pad instruction}

%%% cleanuppad instruction

\subsubsection{Clean-up pad instruction}




\addtolength{\textheight}{-12cm}   
%\bibliography{final_project}{}
%\bibliographystyle{plain}


\end{document}
